\documentclass[12pt]{article}%{{{
\usepackage[margin=1in]{geometry} 
\usepackage{amsmath,amsthm,amssymb}
\usepackage{listings}
\usepackage{xcolor}
\usepackage[section]{algorithm}
\usepackage{algpseudocode}
\usepackage{enumitem}
\usepackage{textcomp,xspace}  %just for type <>
\definecolor{dkgreen}{rgb}{0,0.6,0}
\definecolor{gray}{rgb}{0.5,0.5,0.5}
\definecolor{mauve}{rgb}{0.58,0,0.82}
\lstset{
  language=C++,
  aboveskip=3mm,
  belowskip=3mm,
  numbers=left,
  frame=single,
  showstringspaces=false,
  columns=flexible,
  basicstyle=\ttfamily,
  numberstyle=\color{gray},
  keywordstyle=\color{blue},
  commentstyle=\color{dkgreen},
  stringstyle=\color{mauve},
  backgroundcolor=\color{black!5},
}

\newenvironment{theorem}[2][Theorem]{\begin{trivlist}
\item[\hskip \labelsep {\bfseries #1}\hskip \labelsep {\bfseries #2.}]}{\end{trivlist}}
\newenvironment{lemma}[2][Lemma]{\begin{trivlist}
\item[\hskip \labelsep {\bfseries #1}\hskip \labelsep {\bfseries #2.}]}{\end{trivlist}}
\newenvironment{exercise}[2][Exercise]{\begin{trivlist}
\item[\hskip \labelsep {\bfseries #1}\hskip \labelsep {\bfseries #2.}]}{\end{trivlist}}
\newenvironment{reflection}[2][Reflection]{\begin{trivlist}
\item[\hskip \labelsep {\bfseries #1}\hskip \labelsep {\bfseries #2.}]}{\end{trivlist}}
\newenvironment{proposition}[2][Proposition]{\begin{trivlist}
\item[\hskip \labelsep {\bfseries #1}\hskip \labelsep {\bfseries #2.}]}{\end{trivlist}}
\newenvironment{corollary}[2][Corollary]{\begin{trivlist}
\item[\hskip \labelsep {\bfseries #1}\hskip \labelsep {\bfseries #2.}]}{\end{trivlist}}%}}}

\title{Homework \#2}
\author{Shao Hua, Huang\\
DEE2505 - Data Structures}

\begin{document}
\maketitle
\section{Objective}
\begin{enumerate}
  \item To understand how an implementation of an ADT is used by an application program.
  \item To become familiar with how to implement Linked List.
  \item To understand how to use list in STL library in c++.
\end{enumerate}
\section{Pseudo code}
Declare variable order (CLOCKWISE or COUNTERCLOCKWISE).\\
Write a class Node whose members are id number and letters the person called out.\\
Following are Node class declaration and pseudo codes of the program.\\\\
\begin{lstlisting}%{{{

enum {CLOCKWISE, COUNTERCLOCKWISE} order;
class Node {
  public:
    Node(int id, string letter);
    ~Node(){}
    int getid();
    string getletter();
    void addletter(char, string);
  private:
    int id;
    string letter;
}
\end{lstlisting}%}}}
\begin{algorithm}%{{{
  \caption{main}
  \begin{algorithmic}[1]
    \State $fin.open(argv[1])$
    \State $fout.open(argv[2])$
    \State $list$\textlangle Node\textrangle $mylist$
    \State $list$\textlangle Node\textrangle $result$
    \State $list$\textlangle Node\textrangle::iterator $it$
    \State $order \gets$ CLOCKWISE
    \State get input from $fin$, and push back id number to $mylist$ until negative number received
    \State use iterator to find which person to be the first one, assign to $it$
    \While{get string $str$ in $fin$}
      \State $callTheWord(mylist, it, str)$
      \State $changeOrderIfNeed(str)$
      \State $oneOutFromGame(mylist, result, it)$
    \EndWhile
    \State push the last node in $mylist$ to $result$ list
    \State use iterator to print out id numbers and letters in $result$ list
    \State $fin.close()$
    \State $fout.close()$
  \end{algorithmic}
\end{algorithm}%}}}
\begin{algorithm}%{{{
  \caption{callTheWord(list\textlangle Node\textrangle \&, list\textlangle Node\textrangle::iterator \&, string)}
  \begin{algorithmic}[1]
    \For{$i$ from $0$ to $str.size()-2$}
      \If{$order$ = CLOCKWISE}
        \State add letter $str[i]$ to current node's letter from back
      \ElsIf{$order$ = COUNTERCLOCKWISE}
        \State add letter $str[i]$ to current node's letter from front
      \EndIf
      \State $putItToNextPlace(mylist, it)$
    \EndFor
    \If{$order$ = CLOCKWISE}
      \State add letter $str[str.size()-1]$ to current node's letter from back
    \ElsIf{$order$ = COUNTERCLOCKWISE}
      \State add letter $str[str.size()-1]$ to current node's letter from front 
    \EndIf
  \end{algorithmic}
\end{algorithm}%}}}
\begin{algorithm}%{{{
  \caption{changeOrderIfNeed(string)}
  \begin{algorithmic}[1]
    \If{$str[str.size()-1]$ is not vowel}
      \State $order\gets$ COUNTERCLOCKWISE
    \Else
      \State $order\gets$ CLOCKWISE
    \EndIf
  \end{algorithmic}
\end{algorithm}%}}}
\begin{algorithm}%{{{
  \caption{oneOutFromGame(list\textlangle Node\textrangle \&, list\textlangle Node\textrangle \&, list\textlangle Node\textrangle ::iterator \&)}
  \begin{algorithmic}[1]
    \State copy current node and push back to $result$ list
    \State pop current node by using erase function
    \State $putItToNextPlace(mylist, it)$ 
  \end{algorithmic}
\end{algorithm}%}}}
\begin{algorithm}%{{{
  \caption{putItToNextPlace(list\textlangle Node\textrangle \&, list\textlangle Node\textrangle::iterator \&)}
  \begin{algorithmic}[1]
    \If{$order$ = CLOCKWISE}
      \State let the iterator points to the next node
      \If{the node does not exist, \textit{i.e.} $mylist.end()$}
        \State $it\gets mylist.begin()$
      \EndIf
    \ElsIf{$order$ = COUNTERCLOCKWISE}
      \If{the node is the first node, \textit{i.e.} $mylist.begin()$}
        \State $it\gets mylist.end()$
      \EndIf
      \State let the iterator points to the previous node
    \EndIf
  \end{algorithmic}
\end{algorithm}%}}}
\section{Time complexity analysis}%{{{
Let $n$ be number of participants in this game.\\
Let $c$ be the upper bound letters of each word.
\subsection{main}
Line 7: push all id numbers to the list $\rightarrow O(n)$\\
Line 8: find one number in the list $\rightarrow O(n)$\\
Line 9: get $n-1$ strings in while loop $\rightarrow O(n)$\\
Line 10: see 3.2 $\rightarrow O(c)$\\
Line 11: see 3.3 $\rightarrow O(1)$\\
Line 12: see 3.4 $\rightarrow O(1)$\\
Line 15: print result list $\rightarrow O(n)$\\
Other lines: $O(1)$
\subsection{callTheWord}
Line 1: enter for loop $str.size()-1$ times $\rightarrow O(c)$\\
Line 7: see 3.5 $\rightarrow O(1)$\\
Line 9 to 13: push $str[str.size()-1]$ to the list $\rightarrow O(1)$
\subsection{changeOrderIfNeed}
Line 1 to 5: change order if it needed $\rightarrow O(1)$
\subsection{oneOutFromGame}
Line 1 to 2: push current node to $result$ and erase the same node in $mylist\rightarrow O(1)$\\
Line 3: see 3.5 $\rightarrow O(1)$
\subsection{putItToNextPlace}
Line 1 to 11: enter if to move iterator $\rightarrow O(1)$
\subsection{Total time complexity}
In main, we can see that\\
Line 7 to 8: $O(n)$\\
Line 9 to 12: $O(nc)$\\
Line 15: $O(n)$\\
Other line: $O(1)$\\
Therefore, the total time complexity will be $O(nc)$ where n is total participants and c is upper bound letters of each word.%}}}
\section{Conclusion}
In this homework, I learned how to implement linked list by using list in STL library.\\
List provided easy functions for me to use, and I also wrote a class called Node to store data.\\
While getting each string input, the program is also doing tasks to simulate the game.\\
It is not that difficult to implement the whole program, but it surely takes time and efforts.\\
Besides, I also tried to make my code more understandable by refactoring.\\
It improves my coding skill, and I am looking forward to the next homework.
\end{document}
